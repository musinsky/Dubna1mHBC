\documentclass{slides}

\usepackage{ucs}
\usepackage[utf8x]{inputenc}
\usepackage[russian]{babel}
\usepackage{amsmath}
\usepackage[paperwidth=43cm,paperheight=30cm,left=4cm,right=4cm]{geometry}
\usepackage[hidelinks]{hyperref}

\setlength{\parindent}{4ex}
% \centering

\begin{document}
\bfseries

По инициативе В.И.~Векслера в ЛВЭ была создана и экспонировалась в~пучках
\linebreak $\boldsymbol\pi^{\boldsymbol -}$-мезонов, легких ядер и нейтронов
установка с~1-метровой жидководородной \linebreak пузырьковой камерой
(1964 -- 1992 годы).

В создании камеры принимали участие: отдел водородных камер, криогенный
\linebreak отдел, радиотехнический отдел, механические мастерские, оптическая
группа, \linebreak а~также заводы и институты России (<<Серп и молот>>,
Лыткарино, ГОМЗ, \linebreak ЛИТМО) и Чехословакии. Реализован ряд изобретений.

В 1968 году за создание камеры премия ОИЯИ присуждена А.А.~Белушкиной,
\linebreak В.И.~Векслеру, В.Н.~Виноградову, В.В.~Глаголеву, Л.Б.~Голованову,
Е.И.~Дьячкову, \linebreak А.Г.~Зельдович, Н.К.~Зельдович, Э.В.~Козубскому,
Р.М.~Лебедеву, М.~Малы, \linebreak Б.Д.~Омельченко, Ю.К.~Пилипенко,
В.Ф.~Сиколенко, И.С.~Саитову, В.П.~Сергееву, Е.П.~Устенко, И.В.~Чувило,
Ю.А.~Шишову.

Исследования проводились сотрудничеством более 200~специалистов физических
групп 23~институтов 12~стран. Премия ОИЯИ (1997~год) за цикл исследований
по релятивистской ядерной физике.

\vspace{2ex}
\def\UrlFont{\ttfamily\large}
\center{\url{http://lhe.jinr.ru/1mHBC}}

\end{document}
